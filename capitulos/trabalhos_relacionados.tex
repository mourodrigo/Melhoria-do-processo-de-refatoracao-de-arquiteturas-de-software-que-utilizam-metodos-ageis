\chapter{Trabalhos Relacionados}

~\cite{hofmeister2007general}.

5 métodos de arquitetura de software industriais

1 - Attribute-Driven Design
ADD utiliza da decomposição da regra de negócio (processo) de modulos do sistema para criação do design da arquitetura. Requisitos como atributos de qualidade e requisitos funcionais do módulo já devem estar definidos.
Etapas:
a) definição de diretrizes arquiteturais, momento onde é definido o que é ou não importante nos requisitos minimos do modulo
b) definição de um padrão que atenda as diretrizes
c) instancie modulos e aloque funcionalidades de casos de uso, criando views
d) defina interface e modulos filhos, revisando todo o design para ter certeza que nada foi esquecido

2 - Siemens 4 views
(continuar...)




\cite{pressman2009engenharia}


\cite{babar2013agile}

PAG. 21 - Muitas pessoas envolvidas na área tem lados rigorosamente definidos, defendendo uma incompatibilidade entre a agilidade e arquitetura. 


PAG 29 - Com o surgimento do agir engenheiros se preocuparam em com os métodos ágeis se encaixavam em outros métodos e práticas de engenharia.


O agil é uma resposta do mundo para necessidade de projetos que sejam mais responsivos aos interessados no projeto, mais rapidez no desenvolvimento de funcionalidades ao qual usuários se interessam.


Grande pergunta é: quanto de arquitetura eu devo desenvolver versus quanto eu devo flexibilizar durante o projeto? Como e quando eu devo relatora? Quanto de arquitetura eu devo formalizarem documentos? Deve revisar minha Arquitetura? Se sim, quando?


A flexibilidade dos métodos normais não era pra nada. Ter um bom plano antecipado facilita a predição (desde que os requisitos no mudem muito) E facilita a gestão de grandes times.


Os métodos ágeis por outro lado, rejeitam o planejamento preferem trabalho em 
equipe, comunicação frente-a-frente, Flexibilidade adaptação..


Na opinião do doutor, projetos de sucesso requerem uma mistura das duas estratégias.



PAG 44 - Tal sugere que uma refatoração sistemática pode ajudar a prevenir erosões arquiteturais através da avaliação do designer do software existente antes de chamar novos artefatos no sistema atual.


















pag. 88 - cap. 3 - Refactoring Software Architectures (TODO O CAPÕTULO 3)


PG 63 - Na engenharia de software, mudanças são regras e não exceções.

PG64 - a refatoração é um método de melhorar a estrutura sem alterar o comportamento externo[1]. Introduzindo a refatoração sistemática e os padrões de refatoração é possível ajudar engenheiros de software a prover soluções lidando com necessidades decorrentes da refatoração. E também é possível auxiliar evitando a erosao sistema


PG65 - nenhuma aplicação deve ser construída de única vez mas deve ser pensado em pequenas partes onde a cada iteração um requisito ou pequena parte é definido arquiteturalaumente


PG66 - Como engenheiros podem reconhecer que eles precisa melhorar a estrutura de uma implementação?

A existência de:

- código duplicado,
- métodos dúzias de linhas,
- usos de switchs frequentemente


PG68 - 
Artefatos duplicados - Há uma dificuldade em definir quanta replicação é aceitável ou benéfica, e que tipo de repetições de código são consideradas problemas de arquitetura. Não há uma resposta concreta para isso, depende do contexto onde o principio DRY deve ou não ser aplicado.

Papeis indefinidos - Os números componentes devem explicar as responsabilidades para que os envolvidos no projeto entendam facilmente. As responsabilidades devem ser designados para componentes individuais E não sobre múltiplos componentes. Do contrário esses componentes sofreram. Da mesma maneira um componente deve ter apenas uma responsabilidade e deve fazer bem.


Arquitetura muito complexa - complexidades acidentais levam abstrações desnecessárias. Essas abstrações levam a softwares inexpressivos, Como componentes supérfluos ou não coisas. Arquitetura deve ser simplista da melhor maneira possível.

Tudo centralizado: É necessário evitar· dentro do sistema, arquiteturas hub ou spoke devem ser evitados pois sintetizando tudo em um único ponto de falha.

Utilizar soluções caseiras ao invés de melhores práticas
: É necessário evitar reinventar a roda. A probabilidade E métodos já conhecidos e testados serem melhores que novas práticas é alta.

Arquiteturas muito genéricas: sistema sistema podem vir a ser complexo se exigindo muitas configurações por serem muito genéricos E o excesso de abstração pode aumentar a manutenção.


Estruturas com comportamentos assimétricos:

A simetria é o indicador de uma boa qualidade de arquitetura, existem dois tipos de simetria: simetria de comportamento estrutural. Assimetria de comportamento lidar com as funcionalidades no início de sua atividade, E assegura que todas as ações iniciem e terminem de uma maneira adequada. 


Ciclo de dependência:
Caso haja componentes que dependam uns dos outros, existe a possibilidade de haver impactos na testabilidade que modificabilidade do sistema.

Violações de design:
As violações de design, Como evitar utilizar os padrões corretos do projeto O Utilização de padrões diferentes para atender as soluções dos problemas podem diminuir a expressividade do sistema.

Particionamento inadequado de funcionalidades:
Podem causar complexidade acidental

Dependências desnecessárias:
Para reduzir a complexidade número de dependências deve ser minimizado. Dependências que são pouco utilizadas ou pareçam desnecessários devem ser removidas.

Dependências implícitas:
Quando as implementação do sistema contendo dependências que não são viáveis no modelo arquitetural, estas devem ser removidas pois podem causar obrigações desnecessárias. Implementações com mudanças podem causar erros caso o engenheiro de software não esteja atento a essas dependências implícitas. Um exemplo frequente de dependencias implícitas é o  uso frequente de variáveis globais. Também conhecidas como padrão sigleton.

PG 73- 

1. avaliação de arquitetura:
Identifique os problemas de design, Ou Seja, a habilidade da arquitetura de alcançar os termos de qualidade buscados

2. priorização:
Prioriza As questões determinando a prioridade dos componentes afetados.

3. Seleção:
Para cada problema na lista(iniciando com maior prioridade) Como usar as seguintes atividades:
	3.1 Padrão de refatoração adequado. Neste contexto, apropriado significa que o padrão escolhido resolveu problema interna e externamente
	3.2 Se existe mais de um padrão, escolha O que mais se adapta ao design E que melhor trarão impacto positivo na qualidade.
	3.3 cara esse padrão não exista volto para o design da arquitetura convencional.

4. verificação da qualidade: 
	Para cada aplicação de refatoração verifique Como o sistema foi afetado.
	4.1 abordagem formal: prova com métodos formais que a estruturação não alterou comportamento anterior. Abordagens formais são úteis em sistemas de tempo real.
	4.2 avaliação da arquitetura: façam uma avaliação geral
	

Devido à falta de ferramentas que suportem diretamente a refatoração de arquitetura, ao menos é possível utilizar ferramentas existentes que  auxiliem em parte do processo de refatoração, por que auxiliem Na identificação de problemas de arquitetura na fase de análise.

A retratação de um componente não crítico de arquitetura não deve ser aplicado imediatamente após a data de release. Porém, Se uma refatoração em especifico não é aplicada em uma iteração por algum motivo, está se torna um débito de design e precisa ser resolvida na próxima iteração. No set de scrum, programas de arquitetura que não são verificados na sprint atual Deve ser mantido no backlog.


PG75 - 	Exemplos de padrões de refatoração de arquitetura

Quebrar ciclos de dependencia - 
É possível utilizar técnicas como injeção de dependência ou tentar inverter uma ou mais dependências buscando resolver os ciclos.

Divisão em subsistemas - 

O nível de acoplamento e coesão pode ser utilizado como indicador para dividir ou combinar sub-sistemas.

pg78 - OBSTACULOS NA REFATORAÇÃO DE ARQUITETURA

- Organização e gestao 
Os stakeholders do projeto consideram as novas funcionalidades como sendo as mais importantes.
A premissa de que “a arquitetura de software deve ser feita corretamente no primeiro momento para que não haja problemas futuramente” ignora o fato de que existem constantes mudanças na arquitetura inclusive no momento da criação de novas funcionalidades.
Primeiramente, os engenheiros de software não conhecem todos os requerimentos (pelo menos não com todos os detalhes). 
Além disso o design te todo o processo não funciona. O que torna crescimento passo à passo do sistema como a única alternativa.
Porém o crescimento passo à passo requer uma constante avaliação de qualidade de todos os artefatos de design que também precisam de refatoração de arquitetura.

Um problema comum é que a refatoração de arquitetura apenas pode provar seu valor após a conclusão do projeto. Podendo não haver nenhum retorno do investimento, porem é comprovado que a negligencia de padrões de qualidade é muito mais custosa do que a inserção de verificações de qualidade periódicas.

De acordo com a lei de Conway, a organização dita a arquitetura portanto, má organização leva à uma má arquitetura.


- Processo de desenvolvimento
O processo de relatora
cão precisa ser explicitamente integrado no processo de desenvolvimento geral. Do contrário, o gerenciamento do projeto não planejará recursos suficientes para os objetivos da refatoração.
os stakeholder e testadores devem estar cientes da refatoração para que as mudanças possam ser validadas.

- Tecnologia e ferramentas
Devido à falta de ferramentas que suportem diretamente a engenharia de refatoração de arquitetura, o processo de ver feito manualmente.

- Aplicabilidade
Se a erosão de design for grande ao ponto de que a refatoração resolverá apenas os sintomas e não as causas, e reengenharia ou até a reconstrução pode ser mais adequada e eficiente.



 

pat. 107 - referÍncias do cap. 3 























PAG 161 - A refatoração de arquitetura é uma etapa interativa para implementação da variabilidade (habilidade do software de ser adaptado)

PAG 169 - Uma técnica é usada no memomento apropriado, e não é mais utilizada quando não for mais útil.

Refatoração continua foca na resolução do problema, sendo que as atividades de análise devem preceder as atividades de recuperação, Em casa ideal, a refatoração executado continuamente, Pois é mais fácil fazer pequenas mudanças Durante o processo de desenvolvimento do que fazer uma grande mudança em determinado ponto. Porém, grandes refatorações nem sempre podem ser evitados, [52,58,43]









\cite{hofmeister2007general}
