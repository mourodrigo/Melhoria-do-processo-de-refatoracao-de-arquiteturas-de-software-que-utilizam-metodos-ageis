\chapter{Trabalhos Relacionados}

O livro escrito por \cite{babar2013agile} é uma das referencias que destaca-se por ser uma compilação de diversas publicações à respeito da união entre os métodos ágeis e a arquitetura de software. 
Neste são abordados os diversos pontos relacionados à este tema como a detalhada descrição do conceito de arquitetura de software, métodos e práticas utilizadas, exemplos de documentação aplicados e além disso as vantagens e desvantagens no uso da arquitetura de software durante o processo de desenvolvimento.

Também são descritos diversos conceitos relacionados à metodologia ágil, quais suas práticas, 
conceitos básicos, metodologias mais aplicadas, vantagens na sua utilização e a motivação por trás do surgimento da própria metodologia ágil em si.
Além da diversas abordagens também são apresentadas propostas de adaptações entre os métodos ágeis e a arquitetura de software, onde diversos autores descrevem suas experiências na unificação e uso de ambas as práticas.

O presente trabalho diferencia-se por buscar uma proposta de abordagem que melhore o processo, ponto que não é profundamente discutido no livro. Onde os autores limitam-se à ponderar o equilíbrio entre as opiniões visando o uso da metodologia ágil e arquitetura de software simultaneamente.

No artigo de \cite{yang2016systematic} fora realizado um revisão sistemática de literatura para a coleta de estudos que utilizam a arquitetura de software em conjunto com métodos ágeis.
Neste mesmo estudo procurou-se além de entender as possíveis combinações entre método ágil e arquitetura de software, apresentar incompatibilidades entre as duas metodologias. Foram observados os aspectos de atividades de arquitetura, práticas, custos, benefícios e ferramentas utilizadas.
Observou-se que este artigo diferencia-se do presente trabalho pela abordagem direcionada à refatoração de arquitetura, ponto fundamental aplicado pela metodologia ágil e complexo quando aplicado à arquitetura de software em conjunto desta prática.

