\chapter{Referencial Teórico}
Neste capítulo serão apresentados os principais conceitos e as relações que definem a problemá́tica.
\section{Arquitetura de software}
%AS - O que é
O conceito de Arquitetura de Software surgiu em resposta ao aumento da complexidade dos sistemas de software existentes, este vem sendo agregado por de segmentos como: redes, servidores, bancos de dados, interfaces de usuário, dependências de requisitos, atributos de qualidade de software, atributos de segurança, redundância, replicação e escalabilidade \cite{hanschke2015integrating}.

\cite{bass2007software} apresenta a arquitetura de software como sendo um resultado de um conjunto de decisões de negócio aliadas à decisões técnicas, sendo que este resultado é diretamente influenciado pelo ambiente em que a arquitetura está sendo projetada para ser executada. O autor também afirma que a arquitetura é diretamente influenciada pelos \textit{stakeholders} (proprietário, cliente, desenvolvedor, gestor de projeto, tomadores de decisão do produto entre outros) envolvidos no projeto e interessados na construção do mesmo, diferentes \textit{stakeholders} podem exigir determinados requisitos como: baixo custo de desenvolvimento, grande variedade de funcionalidades, fácil customização, otimização para determinados hardwares. É compreensível que um sistema com o mínimo de requisitos de qualidade (performance, disponibilidade, compatibilidade, uso de recursos e usabilidade para o usuário) deva atender a todos os requisitos de qualidade impostos inclusive pelos \textit{stakeholders} mas, esta será uma árdua tarefa à ser balanceada pelo arquiteto de software no momento do design.

\cite{babar2014making} define a arquitetura de software como:
\begin{itemize}
\item Um conjunto de elementos arquiteturais (ex: informações, processamentos, conexões entre requisitos, modelagens de dados e processos).
\item A forma como estes elementos agem como princípios para guiar  o relacionamento entre elementos e suas propriedades.
\item A análise racional para a escolha de elementos e sua interação com todo o conjunto do sistema.
\end{itemize}  

%AS - Por que é importante?%
A antecipação da arquitetura de software de sistemas pode trazer uma série de vantagens \cite{babar2013agile}, sistemas complexos tipicamente possuem arquiteturas complexas e quando estas forem compostas por partes originadas de sistemas de terceiros podem gerar a necessidade de ajustes e adequações. 

As tendências recentes na descrição da arquitetura de software como um conjunto de decisões de design com diagnóstico das próprias decisões tem destacado a importância da análise racional na realização e descrição das decisões de design tomadas. \cite{babar2009software}

Segundo \cite{hanschke2015integrating}, é importante considerar a realização da documentação do projeto compreendendo os elementos no design de soluções de software efetivas. Este processo compõe definir arquitetura em diversas perspectivas e níveis de abstração, ponderando sobre todas as várias propriedades da arquitetura, assegurando que esta seja concebível com um conjunto de implementações que irão atender os requisitos do sistema, evoluindo e integrando arquiteturas complexas. A realização da arquitetura de software no processo e ciclo de vida de desenvolvimento de software é considerada a etapa mais desafiadora à ser definida pois é considerada como uma solução traduzida em sistema que deve ser projetada antecipadamente da fase de projeto. 

Considerando a arquitetura de software no contexto do ambiente corporativo, existem três razões fundamentais onde pode ser observada a sua importância \cite{bass2007software}:

\begin{enumerate}
    \item \textit{Comunicação entre os stakeholders.} A arquitetura de software representa uma abstração de um sistema que pode ser utilizada por todos os stakeholders como base para a compreensão mútua, negociação, consenso e comunicação. 
    \item \textit{Decisões antecipadas de design.} A arquitetura de software é capaz de apresentar decisões de requisitos de maneira antecipada e estas ações diminuem o peso sobre o desenvolvimento de maneira à antecipar a resolução de impedimentos que podem surgir através de uma análise detalhada do requisito.
    \item \textit{Abstração transferível do sistema.} Os conhecimentos em torno da arquitetura de um sistema e a sua propriedade intelectual costumam ser mantidos sob a guarda de poucos ou até um único envolvido no projeto. A documentação de arquitetura bem estruturada permite a transferência deste conhecimento para outros \textit{stakeholders} além da replicação e transferência para outros projetos.
    
\end{enumerate}
%AS -  dependência em relação aos requisitos 

\cite{babar2013agile} Para compreender e validar um conjunto de requisitos pode ser necessário o uso da modelagem com a compreensão de uma série de soluções de arquitetura sendo que muitas delas podem ser inviáveis devido à determinadas restrições de requisitos. 

\cite{hanschke2015integrating} afirmaram que uma série de esforços já foram realizados aprimorando processos para suportar o design da arquitetura de sistemas complexas, modelagem, refinação e conexão arquitetural de elementos em design detalhado das implementações.  Em relação ao processo, é necessário identificar e capturar os requisitos funcionais e não funcionais do projeto e então realizar a tentativa de definir uma arquitetura que cumpra estes requisitos.

A composição do escopo de recursos e custo de um sistema ao projetar uma arquitetura baseada em um sistema existente pode facilitar em grande parte o processo devido à existencia dos componentes chave que já podem ser facilmente antecipados e compreendidos \cite{babar2013agile}.

Também é compreensível que requisitos possam ser alternados por outros à partir do momento em que seja identificado a impossibilidade de realização do mesmo  \cite{babar2013agile}.  

\cite{hanschke2015integrating} ressalta que na prática e em ambientes reais as soluções arquiteturais são acompanhadas de diversas restrições, existem importantes pontos que devem ser observados como: quais requisitos podem ser atendidos e como eles devem ser atendidos, principalmente tratando-se de requisitos não funcionais.

Um fator crítico na antecipação da arquitetura é o potencial de excesso de planejamento da arquitetura e consequente excesso de investimento em um requisito que pode nunca ser utilizado esta situação pode ser comparada à situações onde existe um baixo investimento em infra-estrutura onde o sistema poderá falhar na execução do serviço corretamente durante a sua execução \cite{grundy2012guest}. 

%AS - impacto da arquitetura nas decisões sobre a fase de desenvolvimento, 

\cite{beck2001manifesto} também revela que um erro crítico é a impossibilidade de adaptar-se ao grande número de mudanças ou re-priorização de requisitos que podem vir à ocorrer com o aumento da experiência dos usuários após a utilização do sistema. A antecipação de design de arquitetura implica na manutenção de alguns requisito gerais, funcionais e não funcionais, para que permaneçam consistentes ao longo de todo o ciclo de vida do projeto para que o relacionamento entre requisitos e arquitetura deve torne-se uma única evolução. 


\subsection{Orientação à objetos}

Na concepção da orientação à objetos, Alan Kay imaginou objetos como recursões do conceito de um computador, sendo cada um desdes objetos responsável pela correta realização de pequenas tarefas próprias sendo este sistema passível de uma melhor compreensão após ser observado em maior escala, funcionando em conjunto com diversos outros objetos integralmente \cite{reenskaug2009dci}. O autor afirma que a Orientação à Objetos é altamente distribuída, sendo de fato um paradigma complexo de delinear em situações em que ocorrem concorrência e paralelismo.

O paradigma da orientação a objetos é um conceito que pode ser aplicado independente da linguagem de programação utilizada \cite{rumbaugh1991object}. O desenvolvimento de softwares utilizando a orientação à baseado em objetos é uma prática recomendada por senso comum entre a comunidade de desenvolvedores. Dentro deste paradigma são apresentados alguns conceitos que compõem a própria orientação à objetos:

\begin{enumerate}
% \item \textit{Comunicação entre os stakeholders.}
\item \textit{Abstração:} É a visão geral de determinados componentes de um sistema com o objetivo de interligá-los à outros componentes, omitindo detalhes específicos, concentrando-se apenas em seu comportamento e interação no sistema como um todo. Segundo \cite{rumbaugh1991object} a abstraçào consiste na percepção dos aspectos fundamentais de uma entidade ou seja, antes de analisar sua implementação, concentrar-se no que o objeto faz e o que é. Através da abstração é possivel obter a liberdade de tomar decisões evitando influenciar em detalhes específicos do objeto antes de sua implementação.

\item \textit{Atributos:} 
Os atributos são informações que compõe o objeto em si, sendo estas características, propriedades estados ou qualidades de algo. As várias instâncias de um objeto podem variar os valores armazenados em cada um dos objetos porém, os nomes dos atribútos em cada uma das classes individualmente é único \cite{rumbaugh1991object}.

\item \textit{Objetos e classes:}  
\cite{rumbaugh1991object} descreve a classe como uma abstração que descreve importantes propriedades de uma aplicação, sendo que os objetos com atributos iguais e mesmo comportamento agrupados em classes e estas, descrevem diversos objetos individuais portanto, um objeto é uma instancia de sua própria que por sua vez tem seu próprio valor para cada atributo compartilhando as operações e nomes de atributos com outras instâncias da mesma classe.

\item \textit{Atributos:} São as informações armazenadas em cada instancia de objeto de uma determinada classe. Segundo \cite{rumbaugh1991object} instâncias diferentes de determinados objetos podem ou não ter valores idênticos. O nome de cada atributo é único no contexto da classe porém não é exclusivo no contexto de todo o sistema.

\item \textit{Herança:} \cite{rumbaugh1991object} define a herança como a troca de informações entre classes desta maneira, uma classe pode ser projetada em seu primeiro momento de abstração em uma classe maior porém, graças a herança é possível definir diversas classes menores com herança de informações de uma classe maior (superclasse) e em seguida adicionar nesta as características exclusivas desta classe em si. 

\item \textit{Generalização e Especialização:} Na definição de \cite{rumbaugh1991object} e \cite{booch2005unified} as generalizações são relacionamentos ou abstrações entre itens genéricos (superclasses) e específicos (subclasses). Enquanto as especializações são classes mais detalhadas que herdam características de classes gerais.

\item \textit{Agregação:} A agregação é a maneira como determinados objetos são associados para que as classes possam representar o sistema como um todo, através dela é possível representar o ligações onde determinados objetos podem conter um ou mais objetos de outras classes como por exemplo, a asa é parte do avião ou o avião tem asas.

\item \textit{Associação:} A associação é compreendida como um relacionamento entre objetos, no caso das associações é importante determinar a quantidade de objetos que pode ser ligados à outros objetos ou instancias uma determinada associação. O intervalo de inteiros que define o tamanho possível ou quantidade de relações é denominada multiplicidade do papel de uma associação, sendo esta expressa com um valor mínimo e máximo (podendo ser iguais), para a separação dos valores é usado o caractere ";". Seguindo este padrão o número de objetos pode ser representado como: um (1), zero ou um (0..1), muitos (0..*) e um ou mais (1..*) \cite{rumbaugh1991object}.


\item \textit{Operações e métodos:}
% to-do
\item \textit{Encapsulamento:}
% to-do



\end{enumerate}

\subsection{Conceitos }

%TO-DO complementar

\subsection{Modelo-visão-controlador}

%TO-DO

\section{Arquitetura de software com métodos ágeis}

% surgimento

Com o decorrer do tempo a comunidade de desenvolvimento de software iniciou uma reação buscando suprir a necessidade de respostas mais rápidas aos stakeholders, sendo mais ativo no desenvolvimento de funcionalidades das quais o usuário realmente percebe propostas de valor e após o surgimento dos métodos ágeis os arquitetos de software questionaram-se sobre o como estes métodos se encaixariam nas práticas de arquitetura de software. \cite{babar2013agile} ressalta que a inflexibilidade dos métodos normais tem um propósito. Obter um planejamento antecipado permite uma ótima predição do projeto desde que os requisitos não mudem com frequência, além de facilitar a coordenação de um grande número de times de desenvolvimento.

% distancia arquitetura e ágil

Segundo \cite{abrahamsson2010agility} o distanciamento entre os métodos ágeis e a arquitetura de software refere-se ao grande tempo desprendido na antecipação de problemas relacionados à arquitetura, enquanto os métodos ágeis procuram ser adaptativos, tomando decisões conforme a percepção de mudança de requisitos. 

Na literatura, vários estudos apontam problemas relacionados a evolução do projeto de software e dificuldades com a arquitetura. \cite{waterman2015much} aponta que negligências de atividades de arquitetura podem acontecer dependendo dos valores buscados na entrega e que podem ser diretamente influenciados pelo tempo e orçamento do projeto. \cite{martini2015investigating} e \cite{bellomo2013study} afirmam que as limitações de tempo e orçamento podem resultar na tomada de decisões rápidas visando a entrega de funcionalidades, o que afeta diretamente o tempo dedicado ao planejamento e desenvolvimento de uma arquitetura adequada.

Um arquiteto de software que projeta uma arquitetura de sistema com datas reais de entrega e prazos apertados irá tomar um determinado conjunto de decisões porém, o mesmo arquiteto ao projetar um sistema similar onde as datas de entrega e prazos podem ser facilmente atendidas, certamente tomará decisões diferentes. Da mesma maneira criar um design de arquitetura com os mesmos requisitos de hardware, recursos e interfaces atuais não tem o mesmo impacto que desenvolver uma arquitetura para um sistema que será entregue e executada daqui a 5 anos \cite{bass2007software}.

Dentro da própria comunidade de pessoas envolvidas em processos de desenvolvimento e design de softwares existem lados rigorosamente definidos  uma incompatibilidade entre a agilidade e arquitetura. Neste contexto é necessário refletir sobre as seguintes perguntas: Quanto de arquitetura eu devo desenvolver versus quanto eu devo flexibilizar durante o projeto? Como e quando eu devo refatorar projetos? Quanto de arquitetura eu devo formalizar em documentos? Devo revisar minha arquitetura? Se sim, quando?
\cite{babar2013agile}

\section{Os métodos ágeis e a arquitetura de software}

%  Valores e princípios dos métodos ágeis versus Métodos ágeis Scrum e XP  versus Arquitetura de software

% enfatizar arquitetura não é valor percebido pelo cliente

\section{Documentação de arquitetura nos métodos ágeis}

\section{Práticas que permitem melhorar a arquitetura nos métodos ágeis}

Em \cite{babar2013agile}, Michael Stal (co-autor do livro) enfatiza a importância da refatoração sistemática do software na prevenção da erosão de arquitetura (situação decorrente da mudança de requisitos do sistema sem a evolução da arquitetura). De acordo com o autor, a refatoração sistemática da arquitetura  permite que o arquiteto de software previna a erosão através da avaliação do software existente antes da inclusão de novos componentes ou alteração dos componentes existentes. Também é recomendada uma abordagem de refatoração de arquitetura adaptável ao método ágil que consiste na identificação do problema, aplicação das técnicas adequadas de refatoração e testes da arquitetura resultante após as alterações desta maneira, a refatoração de arquitetura de software também é combinada com atividades de refatoração de código para a melhor agregação de valor, oferecendo melhores opções para os arquitetos de software na tomada de decisões. Nesta mesma publicação, John Grundy revela que diversos aspectos da arquitetura podem ser explorados utilizando \textit{spikes} (pequenos programas desenvolvidos no intuito de examinar uma possível solução para um requisito) facilitando o processo de refatoração.

Para o efetivo funcionamento da arquitetura de software em conjunto dos métodos ágeis é importante realizar o trabalho arquitetural de maneira clara e sólida explicitando os objetivos das atividades à serem realizadas. O problema com a maioria das definições de requisitos à serem atendidos é a sua própria definição, sendo os autores de \cite{babar2013agile} apresentam três perguntas que ao serem respondidas são fundamentais nesta etapa:

\begin{itemize}
    \item \textit{Por que?} Por que o trabalho de arquitetura está sendo feito? Qual o seu propósito? Qual o benefício proposto?
    \item \textit{Como?} Como este trabalho de arquitetura é realizado? Quais as atividades de arquitetura à serem ralizadas?
    \item \textit{O que?} Quais os objetivos desta arquitetura? Quais os principais componentes?
\end{itemize}

Estas perguntas foram elaboradas pelos autores baseadas na sua vasta experiência e anos de observações em projetos comerciais e discussões com arquitetos especialistas da área, através delas é possível identificar com clareza os requisitos à serem atendidos e desta maneira, avaliar corretamente a prioridade e dificuldade de implementação de cada um destes requisitos..

% to-do complementar mais praticas pg236 do livro

\section{Refatoração de arquitetura}

Tratando-se de sistemas de software, a expansão do mesmo com novas funcionalidades e mudanças sem o devido acompanhamento conduzirá o surgimento de uma arquitetura com falhas que podem ser difíceis de eliminar. 

\cite{babar2013agile} definem que a refatoração é um método de melhorar a estrutura interna do sistema sem alterar o comportamento externo, introduzindo a refatoração sistemática e os padrões de refatoração é possível ajudar engenheiros de software a prover soluções lidando com necessidades decorrentes da refatoração além de auxiliar evitando a erosão sistema.
Nenhuma aplicação deveria ser construída de única vez,  deveria entretanto ser pensada e desenvolvida em pequenas partes onde a cada iteração um requisito ou pequena parte é definido arquiteturalmente e em seguida implementado.

Devido à falta de ferramentas que suportem diretamente a refatoração de arquitetura, é possível utilizar ferramentas existentes que auxiliem ao menos em parte do processo de refatoração como por exemplo, na identificação de problemas de arquitetura na fase de análise ou mediante análise de código já implementado.

Segundo \cite{babar2013agile} arquitetos de software podem reconhecer a necessidade de melhorias de arquitetura quando o código apresentar as seguintes características: 
    \begin{itemize}
        \item Código duplicado
        \item Métodos com muitas de linhas.
        \item Uso de switchs em excesso.
    \end{itemize}

Além das características detalhadas de código que caracterízam as chamadas \textit{architecture smells} (termo utilizado para a presença de "cheiros ruins" na arquitetura, onde há a necessidade de aplicar refatorações para resolver possíveis problemas) ainda existem outros artefatos que podem ser identificados na arquitetura em si que podem caracterizar a necessidade de refatoração.

\begin{description}

\item[Artefatos duplicados:] Há uma dificuldade em definir quanta replicação é aceitável ou benéfica, e que tipo de repetições de código são consideradas problemas de arquitetura. Não há uma resposta concreta para isso, depende do contexto onde o principio DRY \textit{(Don't Repeat Yourself)} deve ou não ser aplicado.

\item[Papeis indefinidos:] O nomes dos componentes devem explicar as responsabilidades para que os envolvidos no projeto entendam facilmente. As responsabilidades devem ser designados para componentes individuais e não sobre múltiplos componentes. Da mesma maneira um componente deve ter apenas uma responsabilidade e deve faze-la bem.

\item[Arquitetura muito complexa:] Complexidades acidentais levam abstrações desnecessárias. Essas abstrações levam a softwares inexpressivos, Como componentes supérfluos ou dependências. Arquitetura deve ser simplista da melhor maneira possível.

\item[Tudo centralizado:] É necessário evitar ao máximo dentro do sistema, arquiteturas \textit{hub} devem ser evitadas pois podem acabar sintetizando tudo em um único ponto de falha.

\item[Utilizar soluções caseiras ao invés de melhores práticas:] É necessário evitar reinventar a roda. A probabilidade de métodos já conhecidos e testados serem melhores que novas práticas é alta.

\item[Arquiteturas muito genéricas:] Sistemas genéricos podem vir a ser complexos se exigirem muitas configurações por serem muito e o excesso de abstração pode aumentar a manutenção.

\item[Estruturas com comportamentos assimétricos:]
A simetria é o indicador de uma boa qualidade de arquitetura, existem dois tipos de simetria: simetria de comportamento e estrutural. Assimetria de comportamento lida com as funcionalidades no início de sua atividade, e assegura que todas as ações iniciem e terminem de uma maneira adequada. A assimetria estrutural garante que todas as questões sejam resolvidas utilizando soluções padronizadas.

\item[Ciclo de dependência:]
Caso haja componentes que dependam uns dos outros, existe a possibilidade de haver impactos na testabilidade e modificabilidade do sistema.

\item[Violações de design:]
As violações de design, como evitar utilizar os padrões corretos do projeto e utilização de padrões diferentes para atender as soluções dos problemas podem diminuir a expressividade do sistema.

\item[Particionamento inadequado de funcionalidades:]
O particionamento inadequado pode causar complexidade adicional não desejada na arquitetura.

\item[Dependências desnecessárias:]
Para reduzir a complexidade número de dependências deve ser minimizado. Dependências que são pouco utilizadas ou pareçam desnecessários devem ser removidas.

\item[Dependências implícitas:]
Quando as implementação do sistema contendo dependências que não são viáveis no modelo arquitetural, estas devem ser removidas pois podem causar obrigações desnecessárias. Implementações com mudanças podem causar erros caso o engenheiro de software não esteja atento a essas dependências implícitas. Um exemplo frequente de dependências implícitas é o uso frequente de variáveis globais. Também conhecidas como padrão sigleton.

\end{description}

A refatoração de um componente não crítico de arquitetura não deve ser aplicado imediatamente após a data de release. Porém, se uma refatoração em especifico não é aplicada em uma iteração por algum motivo, está se torna um débito de design e precisa ser resolvida na próxima iteração, já artefatos de arquitetura que não foram verificados na sprint atual devem ser mantido no backlog para serem analisado em seguida.

A refatoração contínua \cite{beck2000extreme} foca na resolução de problemas acima da análise, mesmo que as atividades de análise precedam as atividades de refatoração. Em um caso ideal, a refatoração é executada continuamente \cite{stamelos2007agile} pois, é mais fácil implementar uma série de pequenas mudanças durante o processo de desenvolvimento do que uma grande mudança. Porém, nem sempre é possível evitar grandes mudanças no processo \cite{lippert2006refactoring}. 


\subsection{Principais obstáculos na refatoração de arquitetura}
\begin{description}
\item[Organização e gestão:] 
Os stakeholders do projeto consideram as novas funcionalidades como sendo as mais importantes.
A premissa de que “a arquitetura de software deve ser feita corretamente no primeiro momento para que não haja problemas futuramente” ignora o fato de que existem constantes mudanças na arquitetura inclusive no momento da criação de novas funcionalidades.
Primeiramente, os engenheiros de software não conhecem todos os requerimentos (pelo menos não com todos os detalhes). 
Além disso o design de todo o processo não funciona. O que torna crescimento passo à passo do sistema como a melhor alternativa porém, o crescimento passo à passo requer uma constante avaliação de qualidade de todos os artefatos de design que também precisam de refatoração de arquitetura. Um problema comum é que a refatoração de arquitetura apenas pode provar seu valor após a conclusão do projeto podendo assim não haver nenhum retorno do investimento à curto prazo porem, é comprovado que a negligência de padrões de qualidade é muito mais custosa do que a inserção de verificações de qualidade periódicas. De acordo com a lei de Conway, a organização dita a arquitetura portanto, uma má organização leva à uma má arquitetura.

\item[Processo de desenvolvimento:]
O processo de refatoração precisa ser explicitamente integrado no processo de desenvolvimento geral. Do contrário, o gerenciamento do projeto não planejará recursos suficientes para os objetivos da refatoração,
os \textit{stakeholders} e testadores devem estar cientes da refatoração para que as mudanças possam ser validadas.

\item[Tecnologia e ferramentas:]
Devido à falta de ferramentas que suportem diretamente a engenharia de refatoração de arquitetura, o processo de ver feito manualmente mediante análise da estrutura atual e avaliação da reimplementação conforme a mudança do requisito de qualidade em questão.

\item[Aplicabilidade:]
Se a erosão de design da arquitetura de software do projeto desenvolvido for grande ao ponto de que a refatoração resolverá apenas os sintomas e não as causas, e reengenharia ou até a reconstrução do projeto pode ser mais adequada e eficiente.
\end{description}

\subsection{Refatoração versus reengenharia de arquitetura}

\section{Trabalhos relacionados}

\subsection{Artigo relacionado 1}

\subsection{Artigo relacionado 2}