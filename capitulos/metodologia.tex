\chapter{Metodologia}

A pesquisa consiste em propor um processo que auxilie na refatoração de arquitetura em projetos de software que utilizam métodos ágeis.

O plano de pesquisa a ser utilizado seguirá orientações de \cite{dias2010developing} através uma abordagem baseada em evidências, esta inclui as seguintes atividades principais:
\begin{enumerate}[(a)]

\item Identificar estudos preliminares: condução de revisão sistemática da literatura com a finalidade de identificar estudos relacionados a refatoração de arquitetura nos métodos ágeis, que tratem de métodos ou práticas utilizados e problemas vivenciados.

\item Proposta da abordagem: proposta de uma abordagem para auxiliar na atividade de refatoração de arquiteturas nos métodos ágeis: definição de atividades a serem desenvolvidas, processo a ser seguido e recomendações de forma de adequação aos princípios ágeis. A proposta será construída baseada nas evidências. 
% to-do obetidas na etapa A (revisão da literatura)

\item Validação da proposta com especialista: Considerando que a proposta apresenta tanto contribuições práticas quanto teórica, a validação da proposta será realizada com especialista da academia e profissionais da indústria. Como representantes da academia serão entrevistados 2 pesquisadores da área de Engenharia de Software e como representantes da indústria serão entrevistados 2 consultores sobre métodos ágeis. Para realização da validação será utilizada a técnica de grupos focados. O planejamento das reuniões, a condução das reuniões e forma de análise dos dados, seguirão as recomendações de \cite{ribeiro2003gruposfocados}.

\item Validação da proposta através da aplicação em uma situação real: realização de uma pesquisação numa empresa que apresenta o contexto de pesquisa estudado. Para realização da pesquisa-ação serão seguidas as recomendações de \cite{dos2008colaboraccao}. As métricas utilizadas para a validação durante o processo de pesquisa-ação serão levantadas com base no acompanhamento de \textit{sprints} realizadas pelo desenvolvimento da empresa.

% , sendo considerados para este levantamento os seguintes relatórios:
    
%   \begin{itemize}
%         \item Diagrama acumulativo de fluxo de tarefas.
%         \item Gráfico de controle para medição de estabilidade de métricas.
%         \item Gráfico de velocidade para histórias implementadas.
%         \item Gráfico burndown para apresentação de alterações de escopo não planejado.
%     \end{itemize}  
    
\end{enumerate}

As validações à serem aplicadas utilizam métricas que já são monitoradas atualmente no ambiente de desenvolvimento da empresa em questão onde a pesquisa-ação será realizada, desta maneira será possível avaliar com maior embasamento o impacto que a aplicação da melhoria proposta tratá ao ambiente de desenvolvimento e consequentemente às métricas de acompanhamento. A figura \ref{fig:estrategia_pesquisa} apresenta a metodologia de pesquisa proposta, que inclui as seguintes atividades principais:
\begin{figure}[ht]
\centering
\includegraphics[scale=0.90]{figuras/estrategia_pesquisa}
\caption{ Estratégia de pesquisa}
% \resizebox{!}{0.3cm}{}
\label{fig:estrategia_pesquisa}
\end{figure}

